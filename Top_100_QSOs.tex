%%%%%%%%%%%%%%%%%%%%%%%%%%%%%%%%%%%%%%%%%%%%%%%%%%%%%%%%%%%%%%%%%%%%%%%%%%%%%%%%%%%%
%%%%%%%%%%%%%%%%%%%%%%%%%%%%%%%%%%%%%%%%%%%%%%%%%%%%%%%%%%%%%%%%%%%%%%%%%%%%%%%%%%%%
%%
%%
%%
%%
%%   Nic Ross 
%%  <npross@roe.ac.uk>
%%                                                                                     
%%%%%%%%%%%%%%%%%%%%%%%%%%%%%%%%%%%%%%%%%%%%%%%%%%%%%%%%%%%%%%%%%%%%%%%%%%%%%%%%%%%%
%%%%%%%%%%%%%%%%%%%%%%%%%%%%%%%%%%%%%%%%%%%%%%%%%%%%%%%%%%%%%%%%%%%%%%%%%%%%%%%%%%%%

\documentclass[usenatbib]{mn2e}
%\documentclass[graybox]{svmult}

%%
%% FOR DOUBLE-SPACE, SINGLE-COLUMN VERSION::
%%
%\documentclass[onecolumn]{emulateapj}
%\linespread{2.6}

\usepackage{graphicx,psfig,fancyhdr,natbib,subfigure}
\usepackage{epsfig, psfig, epsf}
\usepackage{amsmath, cancel}
\usepackage{amssymb}
\usepackage{lscape}
\usepackage{dcolumn}% Align table columns on decimal point
\usepackage{bm}% bold math
\usepackage{hyperref,ifthen}
\usepackage{verbatim}
\usepackage[stable]{footmisc}


%%%%%%%%%%%%%%%%%%%%%%%%%%%%%%%%%%%%%%%%%%%
%       define Journal abbreviations      %
%%%%%%%%%%%%%%%%%%%%%%%%%%%%%%%%%%%%%%%%%%%
\def\nat{Nat} \def\apjl{ApJ~Lett.} \def\apj{ApJ}
\def\apjs{ApJS} \def\aj{AJ} \def\mnras{MNRAS}
\def\prd{Phys.~Rev.~D} \def\prl{Phys.~Rev.~Lett.}
\def\plb{Phys.~Lett.~B} \def\jhep{JHEP}
\def\npbps{NUC.~Phys.~B~Proc.~Suppl.} \def\prep{Phys.~Rep.}
\def\pasp{PASP} \def\aap{Astron.~\&~Astrophys.} \def\araa{ARA\&A}
\newcommand{\preep}[1]{{\tt #1} }

%%%%%%%%%%%%%%%%%%%%%%%%%%%%%%%%%%%%%%%%%%%%%%%%%%%%%
%              define symbols                       %
%%%%%%%%%%%%%%%%%%%%%%%%%%%%%%%%%%%%%%%%%%%%%%%%%%%%%
\def \Mpc {~{\rm Mpc} }
\def \Om {\Omega_0}
\def \Omb {\Omega_{\rm b}}
\def \Omcdm {\Omega_{\rm CDM}}
\def \Omlam {\Omega_{\Lambda}}
\def \Omm {\Omega_{\rm m}}
\def \ho {H_0}
\def \qo {q_0}
\def \lo {\lambda_0}
\def \kms {{\rm ~km~s}^{-1}}
\def \kmsmpc {{\rm ~km~s}^{-1}~{\rm Mpc}^{-1}}
\def \hmpc{~\;h^{-1}~{\rm Mpc}} 
\def \hkpc{\;h^{-1}{\rm kpc}} 
\def \hmpcb{h^{-1}{\rm Mpc}}
\def \dif {{\rm d}}
\def \mlim {m_{\rm l}}
\def \bj {b_{\rm J}}
\def \mb {M_{\rm b_{\rm J}}}
\def \qso {_{\rm QSO}}
\def \lrg {_{\rm LRG}}
\def \gal {_{\rm gal}}
\def \xibar {\bar{\xi}}
\def \xis{\xi(s)}
\def \xisp{\xi(\sigma, \pi)}
\def \Xisig{\Xi(\sigma)}
\def \xir{\xi(r)}
\def \max {_{\rm max}}
\def \gsim { \lower .75ex \hbox{$\sim$} \llap{\raise .27ex \hbox{$>$}} }
\def \lsim { \lower .75ex \hbox{$\sim$} \llap{\raise .27ex \hbox{$<$}} }
\def \deg {^{\circ}}
%\def \sqdeg {\rm deg^{-2}}
\def \deltac {\delta_{\rm c}}
\def \mmin {M_{\rm min}}
\def \mbh  {M_{\rm BH}}
\def \mdh  {M_{\rm DH}}
\def \msun {M_{\odot}}
\def \z {_{\rm z}}
\def \edd {_{\rm Edd}}
\def \lin {_{\rm lin}}
\def \nonlin {_{\rm non-lin}}
\def \wrms {\langle w_{\rm z}^2\rangle^{1/2}}
\def \dc {\delta_{\rm c}}
\def \wp {w_{p}(\sigma)}
\def \PwrSp {\mathcal{P}(k)}
\def \DelSq {$\Delta^{2}(k)$}
\def \WMAP {{\it WMAP \,}}
\def \cobe {{\it COBE }}
\def \COBE {{\it COBE \;}}
\def \HST  {{\it HST \,\,}}
\def \Spitzer  {{\it Spitzer \,}}
\def \ATLAS {VST-AA$\Omega$ {\it ATLAS} }
\def \BEST   {{\tt best} }
\def \TARGET {{\tt target} }
\def \TQSO   {{\tt TARGET\_QSO}}
\def \HIZ    {{\tt TARGET\_HIZ}}
\def \FIRST  {{\tt TARGET\_FIRST}}

\newcommand{\sqdeg}{deg$^{-2}$}
\newcommand{\lya}{Ly$\alpha$\ }
%\newcommand{\lya}{Ly\,$\alpha$\ }
\newcommand{\lyaf}{Ly\,$\alpha$\ forest}
%\newcommand{\eg}{e.g.~}
%\newcommand{\etal}{et~al.~}
\newcommand{\cii}{C\,{\sc ii}\ }
\newcommand{\ciii}{C\,{\sc iii}]\ }
\newcommand{\civ}{C\,{\sc iv}\ }
\newcommand{\SiIV}{Si\,{\sc iv}\ }
\newcommand{\mgii}{Mg\,{\sc ii}\ }
\newcommand{\feii}{Fe\,{\sc ii}\ }
\newcommand{\feiii}{Fe\,{\sc iii}\ }
\newcommand{\caii}{Ca\,{\sc ii}\ }
\newcommand{\halpha}{H\,$\alpha$\ }
\newcommand{\hbeta}{H\,$\beta$\ }
\newcommand{\oi}{[O\,{\sc i}]\ }
\newcommand{\oii}{[O\,{\sc ii}]\ }
\newcommand{\oiii}{[O\,{\sc iii}]\ }
\newcommand{\heii}{[He\,{\sc ii}]\ }
\newcommand{\nv}{N\,{\sc v}\ }
\newcommand{\nev}{[Ne\,{\sc v}]\ }


\begin{document}

\title[Quasar Top 100]
{An Observational Overview of Quasars}

\author[N.P. Ross]
{Nicholas P. Ross$^{1,2}$\\ % \thanks{Email: npross@roe.ac.uk},\\
$^1$Institute for Astronomy, University of Edinburgh, Royal Observatory, Blackford Hill 
 Edinburgh EH9 3HJ, United Kingdom\\
$^2$STFC Ernest Rutherford Fellow\\
Email: npross@roe.ac.uk\\
}

\maketitle

\begin{abstract}
The SDSS-III: Baryon Oscillation Spectroscopic Survey (BOSS) is a 5
year experiment which obtained first light in late 2009. One of the
key goals of the BOSS is to obtain spectra of a sample of
$\sim$150,000 $z>2.2$ quasars, in order to measure baryon acoustic
oscillations (BAO) in the distribution of the \lyaf\ to provide a
percent-level measurement of the expansion history of the Universe.
In this paper, we describe the BOSS quasar target selection algorithms,
and characterize their success, for the first year of BOSS observations
only.  The magnitude limit of the BOSS Year One Quasar survey was set
at $g\leq22.0$ or $r\leq21.85$.  In Year One, the BOSS Quasar Survey
obtained a total of \hbox{54,909} spectra from an area of 878.1
deg$^{2}$ (including $\sim40$~deg$^{2}$ of the SDSS deep stripe
``Stripe 82'') at a mean target density of 62.8 targets deg$^{-2}$.
Of the \hbox{33,556} unique objects with good quality spectra,
\hbox{11,149} had redshifts indicative of stellar spectra, and
\hbox{13,580} we quasars with redshifts of $z\geq2.20$. Over 80\% of
these quasars (\hbox{11,263}) were spectroscopically confirmed for the
first time by the BOSS. The $z\geq2.20$ quasar surface density was
15.46 quasars deg$^{-2}$, with a global efficiency of 26.0\%. However,
in areas targeted at the nominal survey target density of 40 targets
deg$^{-2}$, the efficiency rose to $\sim$35\%, ($\sim$14 $z\geq 2.20$
quasars deg$^{-2}$), with our ``CORE'' selection efficiency being just
over $\sim$50\% ($\sim$10 $z\geq 2.20$ quasars deg$^{-2}$ from 20
targets deg$^{-2}$). Finally, we suggest further methods and new,
non-optical, data that can be utilized, so that as the survey
progresses, BOSS quasar targeting will improve in efficiency.

\end{abstract}

%\keywords{surveys -  quasars: Lyman-$\alpha$ forest, cosmology: classification techniques}
\begin{keywords}
galaxies: clustering -- luminous red galaxies: general -- cosmology: 
observations -- large-scale structure of Universe.
\end{keywords}


%%%%%%%%%%%%%%%%%%%%%%%%%%%%%%%%%%%%%%%%%%%%%%%%%%%%%%%%%%%%%%
%%%%%%%%%%%%%%%%%%%%%%%%%%%%%%%%%%%%%%%%%%%%%%%%%%%%%%%%%%%%%%
%%
%%   SECTION 1  SECTION 1  SECTION 1  SECTION 1  SECTION 1  SECTION 1  
%%   SECTION 1  SECTION 1  SECTION 1  SECTION 1  SECTION 1  SECTION 1  
%%   SECTION 1  SECTION 1  SECTION 1  SECTION 1  SECTION 1  SECTION 1  
%%
%%%%%%%%%%%%%%%%%%%%%%%%%%%%%%%%%%%%%%%%%%%%%%%%%%%%%%%%%%%%%%
%%%%%%%%%%%%%%%%%%%%%%%%%%%%%%%%%%%%%%%%%%%%%%%%%%%%%%%%%%%%%%
\section{Introduction}

\cite{Hunt04}
\citet{Richards06}
\citet{Croom09a}
\citet{Croom09b}
\citet{Degraf10}
\citet{Glikman11}
\citet{Glikman11}

Martin's point: {\it (i)} If I had 2-3 slides to present on QSOs for students, what would those be??; {\it (ii)} a list of key things a decent student should know (about e.g. QSOs). {\it (iii)} (and from me!) what's the hook/the new thing here... {\it (iv)} from Joanne, summer school notes. 

What this paper is:  
General overview of the current state of observational QSO work. 
Empirical and phenomolgy. 

What this paper is not: 
Detailed review article. 
Heavy on interpretation, or the subtitles on many of the measurements/observations. 
A theoretical work of any sort. 

In Table~\ref{tab:defintions} various definitions are given. 
\begin{table*}
    \begin{center}
    \setlength{\tabcolsep}{4pt}
    \begin{tabular}{lll}
      \hline
      \hline
    Term  & Definition & Canonical Reference   \\
      \hline
    Quasar & Very luminous object with large redshift, most likely & \\  
                & being powered by gravitational accretion onto a SMBH & \citet{Schmidt63} \\ 
    SMBH & Supermassive Black hole & \citet{LyndenBell_Rees71}\\
    Unobscured QSO  &  UV/optically continuum and emission lines from the accretion disk and BLR & \\
    Obscured QSO     &  UV/optically emission heavily suppressed &\\
    Type 1 AGN     & Broad (and Narrow) Emission Lines present & \\
    Type 2 AGN     & Broad Emission Lines absent in a flux spectrum & \\
    Compton Thick  & AGN with X-ray emission surpressed by $N_{\rm H} \ge 10^{24}$ cm$^{-2}$ & \\
      Type-11 AGN & Unobscured both in optical and X-ray spectra & \citet{Merloni14} \\
      Type-22 AGN & Obscured both at X-ray and optical wavelengths & \citet{Merloni14} \\
      Type-12 AGN & Optically unobscured but have X-ray spectra (or HR) consistent with $N_{\rm H} > 10^{21.5}$ cm$^{-2}$ & \citet{Merloni14} \\ 
      Type-21 AGN & No X-ray obscuration but no broad lines in the optical spectra or have galaxy optical/UV SED & \citet{Merloni14} \\
      Seyfert 1.9 & broad component is detected only in the $H\alpha$ line  and not in the higher-order Balmer lines & \citet{Osterbrock79}\\
      Seyfert 1.8 &  the broad components are very weak but detectable at $H\beta$ as well as $H\alpha$.\\ 
      Seyfert 1.5 & the strengths of the broad and narrow components in $H\beta$ are compar able. & \\
     \hline
    \hline
    \label{tab:defintions}
    \end{tabular}
    \caption{But not only \citep{Schmidt63} }
    \end{center}
    \end{table*}

\section{General AGN Names}
The parent population of flat-spectrum radio-loud Narrow-Line Seyfert 1 galaxies (http://arxiv.org/abs/1504.02772v1)

\section{Photometry} %% (8+1+1+1 = 11)
\begin{itemize}
    \item{Point-like in the optical}
    \item{Extended}
    \item{Can be (very) blue; UVX}
    \item{Can be red; Reddening, \citep{Krawczyk15}}
    \item{Can be very red; high-$z$}
    \item{Can be very red; Obscured Type 1s \citep{Banerji14}}
    \item{Can be very red; Weirdos...\citep{Ross6}}
    \item{Vary; light curves, DRW etc. \citet{MacLeod10, MacLeod11, MacLeod12, MacLeod12_AAS, MacLeod14}}
\end{itemize}
  
    \subsection{X-ray}    

    \subsection{Infra-red}   
    $\lambda>5\mu$m i.e. dust

    \subsection{Radio} 
    \begin{itemize}
      \item{Kratzer et al. }
      \item{blazars vs. Flat-spectrum radio quasars; e.g. Nakagawa\&Mori, 2013, ApJ, 773, 177 also e.g. Justin Finke talk from AAS (!!)  }
      \end{itemize}


\section{Spectroscopy} %% (9+1+1 = 11)
\begin{itemize}
    \item{``Heavily'' redshifted}
    \item{Broad Emission Lines}
    \item{Narrow Emission Lines}
    \item{Permitted emission lines e.g. \lya, \civ, \mgii }
    \item{Semi-Forbidden emission lines e.g. \ciii } 
    \item{Forbidden emission lines e.g. \oiii, \nev } 
    \item{Absorption Lines. Hydrogen, Lyman-$\alpha$ forest}
    \item{Absorption LInes and Reionization}
    \item{Absorption Lines. Metals}
      \item{Double power-law model?? $F_{\nu} = A \nu^{\alpha} + B \nu^{-\alpha}$  \citet{Kishimoto08}}
\item{Big Blue Bump}
\item{Little Blue Bump}
  \item{Composites; \cite{Francis91, VdB01}}
    \item{We introduce Coronal-Line Forest Active Galactic Nuclei (CLiF AGN), Rose arXiv:1501.02705v1).}
\end{itemize}
    \subsection{Objects with ``weird'' spectra''}
    \begin{itemize}
      \item{ERQs}
      \item{Roig's}
        \item{Post-starburst Quasars}
    \end{itemize}



     \subsection{Dual Black Holes}
     Quasar Pairs??

     \subsection{Broad Absoprtion lines}
     

\section{Lensing}  %% (2)
    \begin{itemize}
      \item{``The Quad''}
      \item{Japanese group'}
    \end{itemize}


\section{The Luminosity Function} %% (3)
The QLF is defined as the number density of quasars per unit
luminosity. It is often described by a double power-law
\citep[][hereafter, R06]{Boyle00,Croom04,Richards06} of the form
\begin{equation}
  \Phi(L, z) = \frac{ \phi_{*}^{(L)} }
                            {  (L/L^{*})^{\alpha}    +  (L/L^{*})^{\beta}  }
 \ \,
 \label{eq:double_powerlaw}
\end{equation}
with a characteristic, or break, luminosity $L_{*}$.  An alternative
definition of this form of the QLF gives the number density of quasars
per unit magnitude,
\begin{equation}
  \Phi(M, z) = \frac{ \phi_{*}^{(M)} }
       { 10^{0.4{(\alpha +1)[M-M^{*}(z)]}}+10^{0.4{(\beta +1)[M-M^{*}(z) ]}} }
 \label{eq:double_powerlaw_mag}
\end{equation}
The dimensions of $\Phi$ differ in the two conventions.  We have
followed R06 such that $\alpha$ describes the faint end QLF slope, and
$\beta$ the bright end slope.  The $\alpha$/$\beta$ convention in some
other works \citep[e.g.,][]{Croom09b} is in the opposite sense from our 
definition. Evolution of the QLF can be encoded in the redshift dependence 
of the break luminosity, $\phi_{*}$, and also potentially in the evolution of
the power-law slopes.

\begin{itemize}
  \item{Double Power-law}
  \item{PLE}
  \item{Evolution of $M^{*}$ and $\Phi^{*}$}
  \item{Comparison of the QLF at $z=1, 2, 5$ vs. the galaxy LF}
  \item{Comparison of the QLF$(z)$ vs. the SFR, sSFR}
\end{itemize}
Implications of the QLF, Soltan and/on the ExGal backgrounds....
What wiggle room is left? Espeically that Xray-Spitzzer cross-correlation measurement...


\section{Quasar Clustering} %% (5)
The 2pt Function; Clustering
\begin{itemize}
    \item{$\sim$ few $\times10^{12} M_{\odot}$ to $z\sim2.5$}
    \item{More-or-less luminosity independent for $L_{rm Bol} \gtrsim10^{45}$ ergs.}
    \item{Colour Dependence in e.g. SDSS = none}
    \item{Colour Dependence with $r-W2>6$ gets v. interesting!!}
    \item{a la obscured vs. unobscured}
    \item{With Radio Loudness}
    \item{With X-ray hardnes?!s}
      \item{DM Haloes of Moderate luminosity AGN}
%%     
\end{itemize}


\section{Morphology} %% (5) 
\begin{itemize}
    \item{At $z\sim0$ \citet{Bahcall97}.}
    \item{At $z\sim0.5$ (Stripe 82 stuff??)}
    \item{At $z\sim1$ (Villforth????)}
    \item{At $z\sim1-2$ }
    \item{At $z\gtrsim2$ ???}
\end{itemize}

\section{Scaling Relations} %% (5)
\begin{itemize}
    \item{redshift $z\sim0$, classical relations, $M-\sigma$}
      \item{ \citet{Faber97}, \citet{Magorrian98}, \citet{Ferrarese00}, \citet{Gebhardt00}, \citet{Tremaine02}, \citet{Gultekin09} }
        %% most cited articles by all of the above...
    \item{High-mass end, e.g. \citet{McConnell11, McConnell13}}
    \item{Pseduo-bulges etc. }
    \item{Pseduo-bulges etc. and correcting for e.g. orientation/inclination effects, 
e.g. Bellovary14 }
    \item{\citet{Tremaine02} and \citet{Novak13}}
    \item{Mullaney's work??}
\end{itemize}

\section{Other Relations} %% (5)
\begin{itemize}
    \item{``Type 2'' QSOs are not (extra) massive enough to produce the difference in e.g. 250$\mu$m SF difference.... Chen, Hickox et al.}

\end{itemize}


\section{Cosmo Backgrounds} %% (3) 
\begin{itemize}
  \item{X-ray}
    \item{Optical}
      \item{CIB (Marco Viero :-)   }
      \item{FIR}
      \item{Gamma-ray, $\gamma$-ray; 80\% of Fermi photons from MW; 10\% from Pooint sources, 10\% background, including the cosmo gamma-ray background. e.g. 100 MeV to 820 GeV. arXiv:1410.3696; Blazars account fo 50\% of the EGB. 
Blazars+Extragalactic Background light (which is what exactly?!) responsible for cutt-off in EGB spectrum e.g. Ajello+15; Blazars plus radio gals plus SFgals account for essentially all the EGB (can put constraints on e.g. DM cross-sections.)}
\item{Fermi actually gives a pretty-to-very good feel for the blazar luminosity function and it's evolution -look this up!!}
\end{itemize}


\section{QSO Timescalesd}
\begin{itemize}
  \item{X-ray}
\end{itemize}



\section{General Arguments} %% (3) 
\begin{itemize}
    \item{Soltan}
    \item{Shankar Continuity??}
    \item{QLF+Clustering equals lifetime?!?}
      \item{$M_{\rm BH} = f \frac{R V^{2}}{G}$}
\end{itemize}



\section{To Be Confirmed} %% (1)
\begin{itemize}
  \item{Emit Gravitational Waves}
\end{itemize}


%%
%% This might be a completely different paper, but... 
%%
\section{Connection to DSFGs} 
Having made a relatively comprehensive list of the observed phenomenology...\\
%%Commander Powell: Talk to the bomb. You have to talk to it, Doolittle. Teach it PHENOMENOLOGY. 
is there a connection to  DSFGs...???\\
\begin{itemize}
  \item{Connection of QSOs to Dusty Star-formaig Galaxies...}
    \item{aka how wrong is the Hopkins 07 picture...}
\end{itemize}



%    \begin{figure}
%      \includegraphics[height=8.0cm,width=8.0cm]
%      {pdf/SDSS_Quasar_Nofz.pdf}
%      \centering
%      \caption[he selection of $z \sim 0.7$ LRGs using the SDSS $riz$-bands]
%              {The selection of $z \sim 0.7$ LRGs using the SDSS $riz$-bands.}
%      \label{fig:fig1}
%    \end{figure}



%    \begin{equation}
%      \label{equ:simple_prob}
%      dP = n \, dV.
%    \end{equation}
    
%    \begin{eqnarray}
%      \xi_{LS}(s) &=& 1 + \left(\frac{N_{rd}}{N} \right)^{2} \frac{DD(s)}{RR(s)} -
%      2   \left( \frac{N_{rd}}{N} \right) \frac{DR(s)}{RR(s)} \\
%      &\equiv&  \frac{DD(s)-2DR(s)+RR(s)}{RR(s)},
%      \label{lseq}
%    \end{eqnarray}

%% http://www-astro.physics.ox.ac.uk/~kmb/latex_colour.html
%{\color{red} ...bit of LaTeX text...}
%an example of making an equation the colour DarkSeaGreen (rather a sophisticated shade, I think) one types the following:
%\begin{equation}
%{\color{DarkSeaGreen} x = \log_{10} (\nu/\rm MHz) }
%\end{equation}


%%%%%%%%%%%%%%%%%%%%%%%%%%%%%%%%%%%%%%%%%%%%%%%%%%%%%%%%%%%%%%%%%%%%
%%%%%%%%%%%%%%%%%%%%%%%%%%%%%%%%%%%%%%%%%%%%%%%%%%%%%%%%%%%%%%%%%%%%
%%%%%%%%%%%%%%%%%%%%%%%%%%%%%%%%%%%%%%%%%%%%%%%%%%%%%%%%%%%%%%%%%%%%

%\bibliographystyle{apj}
\bibliographystyle{mn2e}
\bibliography{../../tester_mnras}

\end{document}